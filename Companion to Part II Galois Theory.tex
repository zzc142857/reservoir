%input preamble
\documentclass[12pt]{article}
%Packages
\usepackage[margin=1in]{geometry} 
\usepackage{amsmath,amsthm,amssymb,scrextend}
\usepackage{fancyhdr}
\pagestyle{fancy}
\usepackage{graphicx}
\usepackage{listings}
\usepackage{xcolor}
\usepackage{float}
 \usepackage{hyperref}
 
 %Colours
 \definecolor{amethyst}{rgb}{0.6, 0.4, 0.8}
\definecolor{midnightblue}{rgb}{0.1, 0.1, 0.44}
\definecolor{darkred}{rgb}{0.55, 0.0, 0.0}

%hyperlink
\hypersetup{
    colorlinks=true,
    linkcolor=darkred,
    filecolor=magenta,      
    urlcolor=blue,
    pdfpagemode=FullScreen,
    }
 
%Symbols2
\newcommand{\interior}[1]{{\kern0pt#1}^{\mathrm{o}}}

%mathcal
\newcommand{\curA}{\mathcal{A}}
\newcommand{\curB}{\mathcal{B}}
\newcommand{\curC}{\mathcal{C}}
\newcommand{\curD}{\mathcal{D}}
\newcommand{\curE}{\mathcal{E}}
\newcommand{\curF}{\mathcal{F}}
\newcommand{\curG}{\mathcal{G}}
\newcommand{\curH}{\mathcal{H}}
\newcommand{\curI}{\mathcal{I}}
\newcommand{\curL}{\mathcal{L}}
\newcommand{\curM}{\mathcal{M}}
\newcommand{\curO}{\mathcal{O}}
\newcommand{\curP}{\mathcal{P}}

%mathbb
\newcommand{\C}{\mathbb{C}}
\newcommand{\E}{\mathbb{E}}
\newcommand{\F}{\mathbb{F}}
\newcommand{\I}{\mathbb{I}}
\newcommand{\N}{\mathbb{N}}
\renewcommand{\P}{\mathbb{P}}
\newcommand{\Q}{\mathbb{Q}}
\newcommand{\R}{\mathbb{R}}
\newcommand{\Z}{\mathbb{Z}}

%text
\newcommand{\Aut}{\text{Aut}}
\newcommand{\cha}{\text{char}}
\newcommand{\disc}{\text{disc}}
\newcommand{\edo}{\text{End}}
\newcommand{\Fr}{\text{Frob}}
\newcommand{\id}{\text{id}}
\newcommand{\im}{\text{im}}
\newcommand{\hcf}{\text{hcf}}
\newcommand{\Hom}{\text{Hom}}
\newcommand{\Gal}{\text{Gal}}
\newcommand{\graph}{\text{Graph}}
\newcommand{\GL}{\text{GL}}
\newcommand{\QR}{\text{QR}}
\newcommand{\QNR}{\text{QNR}}
\newcommand{\Root}{\text{Root}}
\newcommand{\SL}{\text{SL}}
\renewcommand{\span}{\text{span}}
\newcommand{\tr}{\text{tr}}


%Environments
\newenvironment{onquestion}[2][On Question]{\begin{trivlist}
\item[\hskip \labelsep {\bfseries #1}\hskip \labelsep {\bfseries #2.}]}{\end{trivlist}}

\renewcommand{\qed}{\hfill$\blacksquare$}
\let\newproof\proof
\renewenvironment{proof}{\begin{addmargin}[1em]{0em}\begin{newproof}}{\end{newproof}\end{addmargin}\qed}

\newenvironment{question}[2][Question]{\begin{trivlist}
\item[\hskip \labelsep {\bfseries #1}\hskip \labelsep {\bfseries #2.}]}{\end{trivlist}}

\newenvironment{remark}[2][Remark]{\begin{trivlist}
\item[\hskip \labelsep {\bfseries #1}\hskip \labelsep {\bfseries #2.}]}{\end{trivlist}}

\newenvironment{theorem}[2][Theorem]{\begin{trivlist}
\item[\hskip \labelsep {\bfseries #1}\hskip \labelsep {\bfseries #2.}]}{\end{trivlist}}




\begin{document}
\title{Companion to Part II Galois Theory}
\author{zzc}
\maketitle

\lhead{Companion to Part II Galois Theory}
\rhead{\today}

\begin{abstract}
    The material is based on Cambridge Part II lectures on Galois theory
\end{abstract}
\tableofcontents

\section{Field Extensions}
\begin{question}{1}
    What is a field extension? Show that the followings are field extensions: (i)$\R/\Q$ (ii)$\C/\R$ (iii)$\Q(\sqrt{2})/\Q$.

    What is the degree of a field extension? Compute the degrees for the extensions given above.
    State and prove the tower law.
\end{question}

\begin{question}{2}
    Let $L/K$ be a field extension. Let $\alpha \in L$. When is $\alpha$ algebraic over $K$? When is $\alpha$ transcendental over $K$? When is $L$ algebraic over $K$? Let $I_\alpha$ be the set of polynomials $f$ in $K[t]$ such that $f(\alpha) = 0$. Show that $I_\alpha$ is an ideal by considering a homomorphism.

    What is an algebraic number?  What is a transcendental number? Show that $\sqrt[9]{7}$ is algebraic. State the Lindemann-Weierstrass theorem. Show that $e$ and $\pi$ are transcendental.

    Show that any finite extension is algebraic.
\end{question}

\begin{question}{3}
    Let $L/K$ be a field extension, $\alpha \in L$. What is the minimal polynomial of $\alpha$ over $K$? Show that the minimal polynomial exists in general. Show that the minimal polynomial is irreducible. Let $f \in K[t]$ be monic and $f(\alpha) = 0$, show tha $f$ is the minimal polynomial of $\alpha$ if and only if $f$ is irreducible.   

    Find the minimal polynomial of $\alpha = \sqrt[3]{2}$ over (i)$\R$ (ii)$\Q$. 
\end{question}

\begin{question}{4}
    Let $L/K$ be a field extension with $\alpha \in L$. What is the field generated by $\alpha$ over $K$?

    Let $\alpha \in L$ be algebraic over $K$. Show that $K(\alpha)$ is the image of the homomorphism $\phi:K[t] \rightarrow L$ by $f \mapsto f(\alpha)$. Show that $[K(\alpha):K] = \deg(P_\alpha)$, where $P_\alpha$ is the minimal polynomial of $\alpha$. 

    Let $L/K$ be a field extension with $\alpha \in L$. Show that $[K(\alpha):K]$ is finite if and only if $\alpha$ is algebraic over $K$.
\end{question}

\begin{question}{5}
    Let $L/K$ be a field extension, $\alpha_1, \dots, \alpha_n$ are elements in $L$. What is the field generated by $\alpha_1,\dots,\alpha_n$ over $K$? Show that if $\alpha_1,\dots,\alpha_n$ are algebraic, then $K(\alpha_1,\dots,\alpha_n)/K$ is finite.

    Let $L/F/K$ be field extensions. Suppose $F/K$ is finite, show that there exist $\alpha_1, \dots, \alpha_n \in L$ such that $F = K(\alpha_1,\dots,\alpha_n)$.
\end{question}

\begin{question}{6}
    State and prove the Eisenstein's criterion for integers.

    Compute $[\Q(\sqrt{2}): \Q]$ and $[\Q(\sqrt[3]{2}): \Q]$. Find $[\Q(\sqrt{2},\sqrt[3]{2}) : \Q]$ without computing $[\Q(\sqrt{2},\sqrt[3]{2}) : \Q(\sqrt{2})]$ and $[\Q(\sqrt{2},\sqrt[3]{2}) : \Q(\sqrt[3]{2})]$. Show that $\sqrt[3]{2} \not \in \Q(\sqrt{2})$ and $\sqrt{2} \not \in \Q(\sqrt[3]{2})$, hence compute $[\Q(\sqrt{2},\sqrt[3]{2}) : \Q(\sqrt{2})]$ and $[\Q(\sqrt{2},\sqrt[3]{2}) : \Q(\sqrt[3]{2})]$ directly.
\end{question}

\begin{question}{7}
    Let $S \subset \R^2$ and $R \in \R^2$. When is $R$ one-step constructible from $S$? When is $R$ constructible from $S$? 

    Let $S = \{(0,0),(1,0)\}$. Show that every point in $\Z^2$ is constructible from $S$.

    Let $S \subset \R^2$ be finite. What is the field of $S$? Let $C$ be a circle or line constructible by ruler or compass respectively from $S$, show that $C$ can be described by equation $a(x^2 + y^2) + bx + cy + d = 0$ for some $a,b,c,d \in \Q(S)$. Show that if $R$ is one-step constructible from $S$, then $[\Q(S \cup \{R\}) : \Q(S)] = 1$ or $2$. Deduce that if $T$ is a finite subset of $\R^2$ containing $S$ such that every point in $T$ is constructible from $S$, then $[\Q(T) : \Q(S)] = 2^k$ for some integer $k \geq 0$. 

    Show that it is impossible to double the cube, i.e. show that $(\sqrt[3]{2}, 0)$ is not constructible from $\{(0,0),(1,0)\}$.
\end{question}

\begin{question}{8}
    Show that any field homomorphism is injective.
    
    Let $L/K$ and $L'/K$ be field extensions. What is a $K$-homomorphism between $L$ and $L'$? Let $[L:K] = [L':K]$ be finite, show that any $K$-homomorphism between $L$ and $L'$ is an isomorphism.

    Determine $\Aut_\R(\C)$,  $\Aut_\Q(\Q(\sqrt{2}))$, and $\Hom_\Q(\Q(\sqrt[3]{2}),\C)$.     
\end{question}

\begin{question}{9}
    What is a Galois extension? What is the Galois group of a Galois extension? Verify that the Galois group is indeed a group.

    Show that $\Q(\sqrt{2})/\Q$ is Galois. Show that $\Q(\sqrt[3]{2})/\Q$ is not Galois. 
\end{question}

\begin{question}{10}
    Let $L/K$ be a field extension, $f \in K[t]$ be irreducible. Show that there is one-to-one correspondence from $\Root_f(L)$ to $\Hom_K(K[t]/(f),L)$. 
    
    Deduce that $|\Hom_K(K[t]/(f),L)| \leq \deg(f)$. In particular, $(K[t]/(f))/K $ is a Galois extension if and only if $\deg(f) = |\Root_f(K[t]/(f))|$.
\end{question}

\begin{question}{11}
    Let $L/K$ be a field extension, $f \in K[t]$. When does $f$ split over $L$? What does it mean by that $L$ contains all roots of $f$? When is $L$ the splitting field of $f$? Show that the splitting field of $f$ is the smallest field over which $f$ splits.  

    Show that $\C$ is the splitting field of $t^2 + 1 \in \R[t]$. 

    Show that $\Q(\sqrt[3]{2},\omega)$ is the splitting field of $t^3 - 2 \in \Q[t]$, where $\omega = e^{i2\pi/3}$.

    Show that for any $K$ subfield of $\C$ and $f \in K[t]$, $f$ has a splitting field $L \subset \C$.
\end{question}

\begin{question}{12}
    Let $K$ be a field, $f \in K[t]$. Show that there exists a field extension $L/K$ such that $L$ is a splitting field of $f$.

    Show that the splitting field of $f$ is unique up to $K$-isomorphism.

    Show that $\Q(\sqrt{7})$ is the splitting field for both $t^2-7$ and $t^2 + 3t +1/2$ in $\Q[t]$.
\end{question}

\begin{question}{13}
    What is an algebraically closed field? Let $L/K$ be a field extension, when is $L$ an algebraic closure of $K$?

    Show that $L$ is algebraically closed if and only if $E$ is finite extension of $L$ implies $E = L$.

    Show that $\C$ is an algebraic closure of $\R$ but not $\Q$.
\end{question}

\begin{question}{14}
    State Zorn's lemma. Let $R$ be a commutative ring, let $I$ be a proper ideal of $R$, show that there exists a maximal ideal containing $I$.

    Let $K$ be a field. Show that $K$ has an algebraic closure. Show that the algebraic closure of $K$ is unique up to $K$-isomorphism.
\end{question}

\begin{question}{15}
    Let $K$ be a field. When is an irreducible polynomial over $K$ separable? When is a general polynomial over $K$ separable? Let $P,Q \in K[t]$, show that if $Q | P$ and $P$ is separable, then $Q$ is separable.

    Show that any linear polynomial is separable.
\end{question}

\begin{question}{16}
    Let $K$ be a field, $f \in K[t]$. What is the formal derivative $f'$ of $f$? 

    Let $f,g \in K[t]$. Show that $(f + g)' = f' + g'$ and $(fg)' = f'g + fg'$. Let $L$ be the splitting field of $f$, show that $f$ and $g$ have a common root in $L$ if and only if $f$ and $g$ have a common irreducible factor in $K[t]$.

    Let $f \not = 0$ and $L$ be the splitting field of $f$. Show that $f$ has a repeated root in $L$ if and only if $f$ and $f'$ have a common irreducible factor in $K[t]$.
\end{question}

\begin{question}{17}
    Let $K$ be a field. What is the characteristic $\cha(K)$ of $K$? Show that if $\cha(K) > 0$, then $\cha(K)$ is a prime number.

    Let $f \in K[t]$ be irreducible. Show that $f$ is inseparable if and only if $f' = 0$. Hence, show that if $\cha(K) = 0$, then $f$ is separable. Show that if $\cha(K) = p$, then $f$ is in separable if and only if $f \in K[t^p]$.
\end{question}

\begin{question}{18}
    Let $L/K$ be an algebraic extension. Let $\alpha \in L$. When is $\alpha$ separable over $K$? When is $L$ separable over $K$?

    Show that $\Q(\sqrt{2})/\Q$ and $\C/\R$ are separable.

    Let $p$ be a prime number, let $L = \F_p(s)$, where $s$ is a formal variable. Let $K = \F_p(s^p)$. Show that $[L:K] = p$, then show that $L$ is inseparable over $K$.

    Find an element $\alpha \in \Q(\sqrt{2},\sqrt{3})$ such that $\Q(\sqrt{2},\sqrt{3}) = \Q(\alpha)$.
\end{question}

\begin{question}{19}
    Let $L/F/K$ be finite extensions, let $E/K$ be a field extension. Let $\alpha \in L$, show that $|\Hom_K(F(\alpha),E)| \leq [F(\alpha):F] \cdot |\Hom_K(F,E)|$.

    Let $L/K$ and $E/K$ be field extensions. Show that $|\Hom_K(L,E)| \leq [L:K]$. Deduce that $|\Aut_K(L)| \leq [L:K]$.

    Let $L/F/K$ be finite extensions and $E/K$ be field extensions. Suppose $|\Hom_K(L,E)| = [L:K]$, show that $|\Hom_K(F,E)| = [F:K]$. Show that every homomorphism in $\Hom_K(F,E)$ can be extended to a homomorphism in $\Hom_K(L,E)$.
\end{question}   

\begin{question}{19.5}
	Let $f  \in K[t]$, $L$ be a splitting field of $K$, $M$ be a field extension of $K$ over which $f$ splits. Show that there exists a $K$-homomorphism from $L$ to $M$. Deduce that if $M$ is a splitting field of $K$, then $L$ and $M$ are $K$-isomorphic.
\end{question}

\begin{question}{20}
    Let $\psi : L \rightarrow E$ be a field homomorphism, and let $\Tilde{\psi} : L[t] \rightarrow E[t]$ be the induced homomorphism by $\psi$. Let $f \in L[t]$, let $g = \Tilde{\psi}(f) \in E[t]$ be irreducible. Show that if $g$ is separable, then $f$ is separable.

\bigskip

    Let $L/K$ be a finite extension. Show that the followings are equivalent.

    (i) There exists field extension $E/K$ such that $|\Hom_K(L,E)| = [L : K]$.

    (ii) $L/K$ is separable.

    (iii) $L = K(\alpha_1,\dots,\alpha_n)$ for some $\alpha_i \in L$ and the minimal polynomial $P_i$ of $\alpha_i$ over $K$ is separable for all $i$.

    *(iv) $L = K(\alpha_1,\dots,\alpha_n)$ for some $\alpha_i \in L$ and the minimal polynomial $R_i$ of $\alpha_i$ over $K(\alpha_1,\dots,\alpha_{i-1})$ is separable for all $i$. (You may ignore (iv) and show the equivalence of (i), (ii), and (iii).)
\end{question}

\begin{question}{21}
    Let $L$ be a field and $G$ be a finite subgroup of the multiplicative group $L^{\times}$, show that $G$ is cyclic.

    State and prove the primitive element theorem.

    Deduce that any finite extension over a field of characteristic $0$ is simple.
\end{question}

\begin{question}{22}
    Let $L = \F_p(s,u)$, where $s$ and $u$ are formal variables. Let $K = \F_p(s^p,u^p)$. Show that $[L:K] = p^2$, hence show that $L/K$ is not simple. Deduce that $L/K$ is not separable.

    Let $K = \F_2$, show that $L = \F_2[t]/(t^2 + t + 1)$ is a field. Show directly that $L/K$ is separable.

    Let $L/K$ be an algebraic extension of a finite field. Show that $L/K$ is separable.
\end{question}

\begin{question}{23}
    Let $L/K$ be an algebraic extension. When is $L/K$ normal? Show that $\Q(\sqrt[3]{2})/\Q$ is not normal.

    Let $L/F/K$ be finite extensions, $\bar K$ be the algebraic closure of $K$. Show that any $\psi \in \Hom_K(F,\bar K)$ can be extended to $\phi \in \Hom_K(L,\bar K)$.

    Let $L/K$ be a finite extension. Show that $L/K$ is normal if and only if $L/K$ is the splitting field for some $f \in K[t]$.
\end{question}

\begin{question}{23.5}
	Let $L/K$ be a finite extension. Let $V = \{f:L\rightarrow L| f \text{ is } K \text{-linear}\}$. Show that $V$ is a $L$-vector space with dimension $[L:K]$. 
	
	Let $G$ be a finite group and $L$ be a field. Let $\{\sigma_1,\dots,\sigma_n\}$ be distinct homomorphisms $G \rightarrow L^\times$. Show that $\{\sigma_1,\dots,\sigma_n\}$ are $L$-linearly independent.
	
	Deduce that $|\Aut(L/K)| \leq [L:K]$.
\end{question}

\begin{question}{24}
    Let $L/K$ be a finite extension, show that the followings are equivalent:

    (i) $L/K$ is a Galois extension.

    (ii) $L/K$ is separable and normal.

    (iii) $L = K(\alpha_1,\dots,\alpha_n)$ for some $\alpha_i \in L$ such that the minimal polynomial $P_{\alpha_i}$ of $\alpha_i$ splits over $L$ and is separable for all $i$.

    Let $K$ be a field, $f \in K[t]$ be a separable polynomial, $L$ be a splitting field of $f$. Show that $L/K$ is Galois.
\end{question}

\begin{question}{25}
    Let $L/K$ be a field extension, $H$ be a subgroup of $\Aut_K(L)$. What is the fixed field of $H$? Show that the fixed field is indeed a field.

    State and prove Artin's lemma.

    Let $L/K$ be a finite extension. Show that $L/K$ is Galois if and only if $L^H = K$, where $H = \Aut_K(L)$.
\end{question}

\begin{question}{25.1(Artin's lemma in parts)}
	Let $G \leq \Aut(L)$ be a finite subgroup. Let $K = L^G$. Show that $L/K$ is algebraic. 
	Show that $L/K$ is separable. 
	Find the minimal polynomial of $\alpha \in L$ over $K$.
	
	Show that $L/K$ is simple hence finite and thus $[L:K] \leq |G|$.
\end{question}

\begin{question}{26}
    Let $L/K$ be a Galois extension, $K\subset F \subset L$ be an intermediate field. Show that $L/F$ is Galois.

    Establish the Galois correspondence for $L/K$. 

    Let $H$ be a subgroup of $\Aut_K(L)$. Show that $H \lhd \Aut_K(L)$ if and only if $\phi(L^H) = L^H$ for all $\phi \in \Aut_K(L)$. Show that $L^H/K$ is norm al if and only if $L^H/K$ is Galois. Hence, show that $H \lhd \Aut_K(L)$ if and only if $L^H/K$ is normal.

    Suppose $H \lhd \Aut_K(L)$, show that $\frac{\Aut_K(L)}{H} \simeq \Aut_K(L^H)$.
\end{question}

\begin{question}{27}
    Let $K$ be a finite field  with $\cha(K) = p$ and $|K| = q$. Show that $q = p^d$ for some $d \in \N$. Let $f = t^q - t \in \F_p[t]$. Show that $K$ is the splitting field of $f$, deduce that $K$ is exactly all roots of $f$.

    Let $p$ be a prime number, $d$ and $d'$ be natural numbers. Let $q = p^d$ and $q' = p ^{d'}$. Show that there exists a field $K$ of $q$ elements, and $K$ is unique up to isomorphism. Call this field $\F_q$. Show that $\F_q$ can be embedded into $\F_{q'}$ if and only if $d|d'$.
\end{question}

\begin{question}{28}
    Let $p$ be a prime number, show that $\bar \F_p = \bigcup_{n \in \N} \F_{p^n}$.

    What is the Frobenius $\Fr_q$ defined on extension $\F_{q^n}/\F_q$, where $q$ is a power of $p$? Show that $\Fr_q \in \Aut_{\F_q}(\F_{q^n})$. Show that $\Fr_q$ has order $n$. Show that $\F_{q^n}/\F_q$ is Galois and $\Gal(\F_{q^n}/\F_q) \simeq C_n$ is generated by $\Fr_q$.

    Define the Frobenius $\Fr_p$ on $\bar \F_p$. Show that $\F_{p^d}$ is the field fixed by $\Fr_p^d$, where $d \in \N$.
\end{question}

\section{Solutions to Polynomial Equations}
\begin{question}{29}
    What is the $n^{\text{th}}$ cyclotomic extension of a field $K$? Let $L$ be the $n^{\text{th}}$ cyclotomic extension of $K$, show that $\Root_{t^n-1}(L)$ is a cyclic subgroup of $L^\times$.

    Let $n \in \N$, $\cha(K) = 0$ or $0 < \cha(K) \nmid n$. Show that $\Root_{t^n-1}(L) \simeq \Z/n\Z$.

    What is an $n^{\text{th}}$ primitive root of unity of a field $K$?

    Show that for each $n\in \N$, there exists $\phi_n \in \Z[t]$ such that 
    
    (i) $\Pi_{d|n, d<n} \phi_d = t^n-1$ for all $n \in \N$,

    (ii) Given any $n\in \N$, let $K$ be a field such that $\cha(K) = 0$ or $0 < \cha(K) \nmid n$, let $L$ be the $n^{\text{th}}$ cyclotomic extension of $K$, then $\Root_{\phi_n}(L) = \{\text{primitive } n^{\text{th}} \text{ roots of unity over } K \}$.

    \bigskip

    Compute $\phi_n$ for $n = 1,\dots,14$ in $\Q[t]$. Verify that $\phi_{15} = (t^4 + t + 1)(t^4 + t^3 + 1)$ over $\F_2$.
\end{question}

\begin{question}{30}
    Let $n\in \N$, let $K$ be a field such that $\cha(K) = 0$ or $0 < \cha(K) \nmid n$, let $L$ be the $n^{\text{th}}$ cyclotomic extension of $K$. Show that $L/K$ is Galois and $\Gal(L/K)$ can be embedded into $(\Z/n\Z)^\times$. Show that any irreducible factor of $\phi_n$ over $K$ has degree $[L:K]$. Show that $\phi_n$ is irreducible in $K[t]$ if and only if $\Gal(L/K) \simeq (\Z/n\Z)^\times$.

    Show that $\phi_n$ is irreducible over $\Q$. Deduce that if $L$ is the $n^\text{th}$ cyclotomic polynomial over $\Q$, then $\Gal(L/\Q) \simeq (\Z/n\Z)^\times$.

    Show that $\phi_8$ is not irreducible in $\F_7[t]$.
\end{question}

\begin{question}{31}
    What is a cyclic extension? Let $n \in \N$, $K$ be a field such that $\cha(K) = 0$ or $0 < \cha(K) \nmid n$. Let $\lambda \in K$ be non-zero. Let $L$ be the splitting field of $f = t^n - \lambda \in K[t]$. Show that $L$ contains a primitive $n^\text{th}$ root $\mu$ of unity. Show that $L/K(\mu)$ is cyclic and $[L:K(\mu)]$ is a factor of $n$. Show that $[L:K(\mu)] = n$ if and only if $f$ is irreducible over $K(\mu)$. 

    Show that $\Q(\sqrt[4]{2},i)/\Q(i)$ is a cyclic extension of degree $4$.
\end{question}

\begin{question}{32}
    What is a Kummer extension?

    Let $L/K$ be a field extension. Show that $\Hom_K(L,L)$ is linearly independent over $L$.

    Let $n \in \N$. Let $K$ be a field such that $\cha(K) = 0$ or $0 < \cha(K) \nmid n$, and $K$ contains a primitive $n^\text{th}$ root of unity. Let $L/K$ be a cyclic extension of order $n$. Show that $L/K$ is a Kummer extension.

    Let $\mu$ be a third primitive root of unity. Show that $\Q(\mu)/\Q$ is a cyclotomic extension of degree $2$ and $\Q(\mu,\sqrt[3]{2})/\Q$ is a Kummer extension of degree $3$.
\end{question}

\begin{question}{33}
    What is a radical extension? Let $f \in K[t]$ for some field $K$, what does it mean by $f$ is soluble by radicals?

    Let $L/K$ be a Galois extension, $\gamma \in L$, $\cha(K) = 0$, let $F$ be the splitting field of $t^n - \gamma$ over $L$. Show that there exists an extension $E/F$ such that $E/K$ is Galois and $E/L$ can be written as a finite sequence of cyclotomic or Kummer extensions.

    Let $\cha(K) = 0$, show that the following statements are equivalent:

    (i) Let $L/K$ be a radical extension, then there exists extension $E/L$ such that $E/K$ is Galois and $E/K$ can be written as a finite sequence of cyclotomic or Kummer extensions.

    (ii) Let $L/K$ be extension that can be written as a finite sequence of cyclotomic or Kummer extensions, then there exists extension $E/L$ such that $E/K$ is Galois and $E/K$ can be written as a finite sequence of cyclotomic or Kummer extensions.

    Show the above statements are true.
\end{question}

\begin{question}{34}
    Let $G$ be a finite group. When is $G$ soluble? Show that any finite abelian group is soluble. Show that $S_3$ and $S_4$ are soluble.

    Let $G$ be a soluble group, show that any subgroup of $G$ is soluble. Let $A \lhd G$, show that $G$ is soluble if and only if both $A$ and $G/A$ are soluble. Show that $S_n$ for $n \geq 5$ are not soluble.
\end{question}

\begin{question}{35}
    What is a soluble extension? Let $L/K$ be a Galois extension, show that $L/K$ is soluble if and only if $\Gal(L/K)$ is soluble.

    Let $K$ be a field with $\cha(K) = 0$, let $L/K$ be a radical extension. Show that $L/K$ is soluble.
\end{question}

\begin{question}{36}
    Let $f \in K[t]$ of degree $n$ has no repeated roots, $\cha(K) = 0$. Let $L$ be the splitting field of $f$. Show that $L/K$ is Galois. Show that $\Gal(L/K)$ can be embedded in $S_n$.

    Let $f = (t^2-2)(t^2 - 3) \in \Q[t]$. Let $L$ be the splitting field of $f$. Find $\Gal(L/\Q)$.
\end{question}

\begin{question}{37}
    Let $p$ be a prime, $\sigma \in S_p$ have order $p$. Show that $\sigma$ is a $p$-cycle.

    Let $f \in \Q[t]$ be irreducible with degree $p$ prime. Let $L \in \C$ be the splitting field of $f$. Show that $f$ has $p$ distinct roots and $L/\Q$ is Galois. Suppose exact $p-2$ roots of $f$ are real, show that $\Gal(L/\Q) = S_p$.

    Show that $f = t^5 -4t + 2 \in \Q[t]$ is not soluble.
\end{question}

\begin{question}{38}
    Let $K$ be a field, $L = K(x_1,\dots,x_n)$ be the field of rational functions of $n$ variables. What is the field of symmetric rational functions in $L$? What are the elementary symmetric polynomials? Show that the elementary symmetric polynomials are in the field of symmetric rational functions.

    Let $K$ be a field, $L = K(x_1,\dots,x_n)$, $F = L^{S_n}$. Show that $L$ is the splitting field of $f = t^n - e_1 t^{n-1} + \dots + (-1)^n e_n \in F$. Show that $L/F$ is Galois and $\Gal(L/F) = S_n$. Show that $F = K(e_1,\dots,e_n)$.
\end{question}

\begin{question}{39}
    What is the general polynomial over a field $K$? When can the general polynomial $f$ of degree $n$ over $K$ be solved by radicals? Show that the general polynomial is irreducible.

    Show that the general polynomial of degree $n \geq 5$ over $K$ with $\cha(K) = 0$ cannot be solved by radicals.
    
\end{question}
    
\begin{question}{40}
    Let $K$ be a field of $\cha(K) = 0$. 
    
    Show that any soluble extension over $K$ is a radical extension. 
    
    Let $h \in K[t]$, and $L$ be the splitting field of $h$. Show that $h$ can be solved by radicals if and only if $\Gal(L/K)$ is soluble. 

    Let $g,h \in K[t]$. Show that if $g$ and $h$ are soluble by radicals, then $gh$ is soluble by radicals.
    
    Show that if $\deg(h) \leq 4$, then $h$ is soluble by radicals.   
\end{question}

\section{Computational Techniques}
Let $L = \Q(x_1,\dots,x_n)$ be the field of rational functions, $F = \Q(e_1,\dots,e_n) \subset L$ be the field of symmetric rational functions. Let $B = \Z[x_1,\dots,x_n]$ and $A = \Z[e_1,\dots,e_n]$.

\begin{question}{41}
     Construct the set $M$ of (monic) monomials in $x_1,\dots,x_n$. Define a total order $<$ on $M$. Check that $(M,<)$ is indeed a total order by checking that $<$ is irreflexive, transitive and trichotomous. Show that $(M,<)$ is well-ordered. Let $m,m_1,m_2 \in M$, show that $m_1 < m_2$ and  implies that $m m_1 < m m_2$. Deduce that $m_1 \leq m_2$ and  implies that $m m_1 \leq m m_2$.
    
    Let $R$ be an integral domain. Let $f, g\in R[x_1,\dots,x_n]$, denote $M_f$ the set of monomials contained in $f$. Show that $M_{fg} = M_f M_g$. Show that $\max M_{fg} = \max M_f \max M_g$.

    Let $E$ be the set of monomials in elementary symmetric polynomials $e_1,\dots,e_n$. Show that $\theta : E \rightarrow M$ by $f \mapsto \max M_f$ is a bijection.
    
    Show that $R[e_1,\dots,e_n]$ is the set of symmetric polynomials in $R[x_1,\dots,x_n]$.

    Deduce that $\Q(e_1,\dots,e_n) \cap \Z[x_1,\dots,x_n] = \Z[e_1,\dots,e_n]$.
\end{question}

\begin{question}{42}
    For $\sigma \in S_n$, let $R_\sigma = t -x_{\sigma(1)}u_1 - \dots -x_{\sigma(n)} u_n = t - \Sigma_{i = 1}^n x_{\sigma(i)} u_i \in B[u_1,\dots,u_n,t]$.
    Let $R = \Pi_{\sigma \in S_n} R_\sigma$. Show that $R \in A[u_1,\dots,u_n,t]$.
\end{question}

\begin{remark}{(Irreducibility Tests)}
    How do we determine whether one polynomial is irreducible. We list down a few methods.

    1. Gauss's lemma, Eisenstein's criterion.

    2. For $f$ cubic, it suffices to consider whether $f$ has linear factors.

    3. Reduction modulo $p$. Let $f \in \Z[X]$, if $f$ is irreducible modulo $p$, then $f$ is irreducible over $\Z$.
    In particular, in the case of modulo $2$, the only irreducible quadratic polynomial is $x^2 + x + 1$.

    4. Let $f \in K[X]$ and $\alpha$ be a root of $f$, then $f$ is irreducible if and only if $[K(\alpha) : K] = \deg f$.
    Examples: suppose we know that $i + \sqrt{2}$ is a root of $X^4 - 2X^2 + 9$, then in order to determine whether $X^4 - 2X^2 + 9$ is irreducible over $\Q$, we may consider the degree of the extension $\Q(i + \sqrt{2})/\Q$. Then we note that $\Q(i + \sqrt{2}) = \Q(i,\sqrt{2})$, so the degree of $\Q(i + \sqrt{2})/\Q$ equal $4$ by tower law, problem solved. A similar example is to show that $X^4 - X^2 + 1$ given that we know it has a root $e^{i\frac{5\pi}{6}}$.
\end{remark}

\begin{remark}{(Find inverses in a special case)}
    Let $f \in \Q[X]$ be irreducible, and let $L = \Q[X]/(f) = \Q(\alpha)$ be a field extension of $\Q$.

    Suppose we want to find the inverse of $g(\alpha)$ in $L$. We can break $g$ into linear factors $X - b$ and try to find inverses of $\alpha - b$. This can be done by dividing $f$ through $X-b$. This works because the remainder of $X-b$ is a constant, so is a unit.
\end{remark}

\begin{question}{(Square root lemma)}

	Let $a \not = b \in \Q$, show that $\Q(\sqrt{a}, \sqrt{b}) = \Q(\sqrt{a} + \sqrt{b})$. 
	
	Deduce that $\Q(e^{i\pi/3})/\Q$ has degree $4$. 
\end{question}
\end{document}