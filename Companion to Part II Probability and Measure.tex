%input preamble
\documentclass[12pt]{article}
%Packages
\usepackage[margin=1in]{geometry} 
\usepackage{amsmath,amsthm,amssymb,scrextend}
\usepackage{fancyhdr}
\pagestyle{fancy}
\usepackage{graphicx}
\usepackage{listings}
\usepackage{xcolor}
\usepackage{float}
 \usepackage{hyperref}
 
 %Colours
 \definecolor{amethyst}{rgb}{0.6, 0.4, 0.8}
\definecolor{midnightblue}{rgb}{0.1, 0.1, 0.44}
\definecolor{darkred}{rgb}{0.55, 0.0, 0.0}

%hyperlink
\hypersetup{
    colorlinks=true,
    linkcolor=darkred,
    filecolor=magenta,      
    urlcolor=blue,
    pdfpagemode=FullScreen,
    }
 
%Symbols2
\newcommand{\interior}[1]{{\kern0pt#1}^{\mathrm{o}}}

%mathcal
\newcommand{\curA}{\mathcal{A}}
\newcommand{\curB}{\mathcal{B}}
\newcommand{\curC}{\mathcal{C}}
\newcommand{\curD}{\mathcal{D}}
\newcommand{\curE}{\mathcal{E}}
\newcommand{\curF}{\mathcal{F}}
\newcommand{\curG}{\mathcal{G}}
\newcommand{\curH}{\mathcal{H}}
\newcommand{\curI}{\mathcal{I}}
\newcommand{\curL}{\mathcal{L}}
\newcommand{\curM}{\mathcal{M}}
\newcommand{\curO}{\mathcal{O}}
\newcommand{\curP}{\mathcal{P}}

%mathbb
\newcommand{\C}{\mathbb{C}}
\newcommand{\E}{\mathbb{E}}
\newcommand{\F}{\mathbb{F}}
\newcommand{\I}{\mathbb{I}}
\newcommand{\N}{\mathbb{N}}
\renewcommand{\P}{\mathbb{P}}
\newcommand{\Q}{\mathbb{Q}}
\newcommand{\R}{\mathbb{R}}
\newcommand{\Z}{\mathbb{Z}}

%text
\newcommand{\Aut}{\text{Aut}}
\newcommand{\cha}{\text{char}}
\newcommand{\disc}{\text{disc}}
\newcommand{\edo}{\text{End}}
\newcommand{\Fr}{\text{Frob}}
\newcommand{\id}{\text{id}}
\newcommand{\im}{\text{im}}
\newcommand{\hcf}{\text{hcf}}
\newcommand{\Hom}{\text{Hom}}
\newcommand{\Gal}{\text{Gal}}
\newcommand{\graph}{\text{Graph}}
\newcommand{\GL}{\text{GL}}
\newcommand{\QR}{\text{QR}}
\newcommand{\QNR}{\text{QNR}}
\newcommand{\Root}{\text{Root}}
\newcommand{\SL}{\text{SL}}
\renewcommand{\span}{\text{span}}
\newcommand{\tr}{\text{tr}}


%Environments
\newenvironment{onquestion}[2][On Question]{\begin{trivlist}
\item[\hskip \labelsep {\bfseries #1}\hskip \labelsep {\bfseries #2.}]}{\end{trivlist}}

\renewcommand{\qed}{\hfill$\blacksquare$}
\let\newproof\proof
\renewenvironment{proof}{\begin{addmargin}[1em]{0em}\begin{newproof}}{\end{newproof}\end{addmargin}\qed}

\newenvironment{question}[2][Question]{\begin{trivlist}
\item[\hskip \labelsep {\bfseries #1}\hskip \labelsep {\bfseries #2.}]}{\end{trivlist}}

\newenvironment{remark}[2][Remark]{\begin{trivlist}
\item[\hskip \labelsep {\bfseries #1}\hskip \labelsep {\bfseries #2.}]}{\end{trivlist}}

\newenvironment{theorem}[2][Theorem]{\begin{trivlist}
\item[\hskip \labelsep {\bfseries #1}\hskip \labelsep {\bfseries #2.}]}{\end{trivlist}}




\begin{document}
\title{Companion to Part II Probability and Measure}
\author{zzc}
\maketitle


\lhead{Companion to Part II Probability and Measure}
\rhead{\today}

\begin{abstract}
    Here are some bookwork questions based on the lecture notes by J. R. Norris on the course titled Probability of Measure in Cambridge. Starred questions are those that I do not know the answers.
\end{abstract}
\tableofcontents


 \section{Measures}
\begin{question}{1}
Let $E$ be a set. What is a $\sigma$-algebra $\curE$ on $E$? What is a set function? What is a measure? What is a measurable space? What is a measure space?

What is a mass function on $E$? Show that if $E$ is countable with $\curE = 2^E$, then there is a one-to-one correspondence between measures on $(E,\curE)$ and mass functions on $E$.
\end{question}

\begin{question}{2}
Let $\curA \subset 2^E$. What is the $\sigma$-algebra $\sigma(\curA)$ generated by $\curA$? Show that $\sigma(\curA)$ is indeed a $\sigma$-algebra.

Let $\curA \subset 2^E$.  When is $\curA$ a $\pi$-system on $E$? When is $\curA$ a $d$-system on $E$? Show that a $\pi$-system that is a $d$-system is a $\sigma$-algebra. 

Let $\curD$ be a $d$-system on set $E$. Let $\curA$ be a subset of $\curD$. Show that $\curD' = \{B\in \curD: B\cap A \in \curD \text{ for all } A\in \curA\}$ is a $d$-system. State and prove Dynkin's $\pi$-system lemma.

Let $\mu_1$, $\mu_2$ be two finite measures on $(E,\curE)$ and $\mu_1(E) = \mu_2(E)$. Suppose $\mu_1$ and $\mu_2$ agree on a generating $\pi$-system $\curA$ of $\curE$, show that $\mu_1 = \mu_2$ everywhere on $\curE$. 

Suppose instead $\mu_1$ and $\mu_2$ are $\sigma$-finite measures such that $E \subset \bigcup_n E_n$ for some disjoint sequence $(E_n)$ in $\curA$, and $\mu_1(E_n) = \mu_2(E_n)$ for all $n$, does the result still hold? Justify your answers.

\end{question}

\begin{question}{3}
(Alternatively, state and prove the Caratheodory extension theorem.)

Let $E$ be a set. Explain a ring of subsets on $E$ and an algebra of subsets on $E$. Show that a countably additive set function on a ring is increasing and countably subadditive.

Let $\curA$ be a ring on $E$ and $\mu : \curA \rightarrow [0,\infty]$ be a countably additive set function. Define the outer measure $\mu^*$ on $E$ and the collection $\curM$ of $\mu^*$-measurable sets on $E$. Show that $\mu^*$ is a countably subadditive increasing set function.

Show that $\mu^*$ agrees with $\mu$ on $\curA$. Then, show that $\curA \subset \curM$.

Show that $\curM$ is an algebra. Hence, show that $\curM$ is a $\sigma$-algebra. Deduce that $\mu^*$ is a measure restricted to the $\sigma$-algebra generated by $\curA$.
\end{question}

\begin{question}{4}
Let $E$ be a Hausdorff space. Explain a Borel $\sigma$-algebra on $E$. Let $\curB$ be the Borel $\sigma$-algebra on $\R$ with the usual topology. Show that $\curB$ is generated by the open intervals in $\R$.

Let $\curA$ be the collection of all finite unions of disjoint intervals of form $(a,b]$. Show that $\curA$ is a ring on $\R$. Show that $\curA$ generates $\curB$ as a $\sigma$-algebra.

Define $\mu(A) = \Sigma_{i=1}^n (b_i - a_i)$ for $A = (a_1,b_1] \cup \dots \cup (a_n,b_n]$ in $\curA$. Show that $\mu$ is well-defined and additive on $\curA$.

 Show that  $\mu$ is countably additive.  Hence, show that there exists a unique Borel measure $\mu$ on $\R$ such that $\mu((a,b]) = b-a$ for every interval $(a,b] \subset \R$. Deduce that $\mu$ is translational invariant.

Is every subset of $\R$ a Borel set? Justify your answer. 
\end{question}

\begin{question}{5}
Let $(\Omega, \curF,\P)$ be a probability space. Explain when a sequence $(\curA_n)$ of sub-$\sigma$-algebras of $\curF$ is independent.

Let $\curA_1$ and $\curA_2$ be two $\pi$-systems in $\curF$. Show that if $\P(A_1 \cap A_2) = \P(A_1) \P(A_2)$ for all $A_1 \in \curA_1$ and $A_2 \in \curA_2$, then $\sigma(\curA_1)$ and $\sigma(\curA_2)$ are independent. Extend the result to the case of $n$ $\pi$-systems.

State and prove the first and second Borel-Cantelli lemmas.
\end{question}

\begin{question}{1.1}
	Let $B \subset \R$ be a Borel set of finite measure. Let $\curA$ be the ring on $\R$ generated by intervals of form $(a,b]$. Show that for all $\epsilon > 0$, there exists $A \in \curA$ such that $\mu(A \triangle B) < \epsilon$.
\end{question}

\begin{question}{1.2}
Let $(E,\curE,\mu)$ be a measure space. What is a null set in $E$? What is the completion $\curE^\mu$ of $\curE$? What is the Caratheodory $\sigma$-algebra $\curM$ extending $\mu$? Show that $\curE^\mu$ is a $\sigma$-algebra to which $\mu$ can be extended. Show that $\curE^\mu = \curM$.
\end{question}
\section{Measurable functions and random variables}
\begin{question}{6}
Let $f$ be a non-negative function on measurable space $(E,\curE)$.  Let $\curD = \{B \in \curB: f^{-1}(B) \in \curE\}$, show that $\curD$ is a $\sigma$-algebra.
Hence, show that $f$ is measurable if either of the following holds:

(i)$f^{-1}((a,\infty]) \in \curE$ for every $a$ positive.

(ii)$f^{-1}([a,\infty]) \in \curE$ for every $a$ positive. 

(iii)$f^{-1}([0,a]) \in \curE$ for every $a$ positive. 
\bigskip

Let $(f_n)$ be a sequence of non-negative measurable functions on $E$. Show that the following functions are measurable:

(i) $f_1 +f_2$ \,
(ii) $f_1 f_2$ \,
(iii) $\inf f_n$ \,
(iv) $\sup f_n$ \,
(v) $\liminf f_n$ \,
(vi) $\limsup f_n$.
\end{question}

\begin{question}{7}
State and prove the monotone class theorem.
\end{question}

\begin{question}{8}
Let $(E,\curE,\mu)$ be a measure space and $(G, \curG)$ be a measurable space. Let $f:E\rightarrow G$ be a measurable function. What is the image measure induced by $f$?

Let $g : \R \rightarrow \R$ be a non-constant, right continuous and increasing function. Set $g(\pm \infty) = \lim_{x\rightarrow \pm \infty} g(x)$ respectively, show that $g(-\infty) < g(\infty)$. Let $I = (g(-\infty),g(+ \infty))$. Define $J_x = \{y \in \R : x \leq g(y)\}$, show that $J$ is a proper subset of $\R$.  Define $f(x) = \inf J_x$, show that $J_x = [f(x),\infty)$. Show that $f: I \rightarrow \R$ left-continuous and increasing such that for all $x \in I$ and $y \in \R$, $f(x) \leq y$ if and only if $x \leq g(y)$.
\end{question}

\begin{question}{9}
Let $g : \R\rightarrow \R$ be a non-constant, increasing, right-continuous functions. Show that there exists unique Radon measure $dg$ on $\R$ such that for every $a<b\in \R$, $dg((a,b]) = g(b) - g(a)$.

Let $\nu$ be a non-zero Radon measure on $\R$, show that there exists a non-constant, increasing, right-continuous function $g:\R\rightarrow \R$ such that $\nu = dg$.
\end{question}

\begin{question}{8}
    Let $(\Omega,\curF,\P)$ be a probability space and $(E,\curE)$ be a measurable space. What is a random variable $X$ in $E$? What is the law or distribution of $X$?
    
    Let $(E,\curE) = (\R,\curB)$. What is the distribution function of $X$? Show that the distribution function determines the distribution.
    
    When is a function $F : \R \rightarrow [0,1]$ a distribution function? Let $(\Omega,\curF,\P)$ be the standard probability space on the unit interval. Given distribution function $F$, construct a random variable on $\Omega$ such that $F = F_X$.
\end{question}

\begin{question}{9}
    Let $X_1$, $X_2$, $Y$ be independent random variables. Show that $\sigma(X_1+X_2) \subset \sigma(X_1,X_2)$. Deduce that $X_1+X_2$ and $Y$ are independent. Let $G_1$ and $G_2$ be Borel functions, show that $G_1 \circ X_1$ and $G_2 \circ X_2$ are independent.

    Let $(X_n)$ be a sequence of independent variables. Let $I$ and $J$ be disjoint subsets of $\N$. *Show that a convergent linear combination of $\{X_i\}_{i\in I}$ and a convergent linear combination of $\{X_j\}_{j\in J}$ are independent.

     By considering the binary expansion of numbers in $[0,1]$, show that there exists a sequence of independent Bernoulli random variables on the same space.

    Let $(F_n)$ be a sequence of distribution functions. Show that there exists a sequence of independent random variables $(X_n)$ on the same space such that $X_n$ has distribution function $F_n$ for all $n$.
\end{question}

\begin{question}{10}
    (a) Let $(E,\curE,\mu)$ be a measure space. Let $(f_n)$ be a sequence of measurable functions on $E$. Let $f:E \rightarrow \R$. Explain when (i)$f_n\rightarrow f$ almost everywhere; (ii)$f_n\rightarrow f$ in measure. Prove the following statements:

    (i) If $\mu$ is finite and $f_n \rightarrow 0$ a.e., then $f_n \rightarrow 0$ in measure.

    (ii) If $f_n \rightarrow 0$ in measure, then $f_n$ has a subsequence $f_{n_k} \rightarrow 0$ a.e.. \bigskip
    
    (b) Let $X$ and $X_n$ be real random variables for all $n\in \N$. Explain when $X_n \rightarrow X$ in distribution.

    Suppose $X$, $(X_n)$ are on the same probability space $(\Omega,\curF,\P)$. Show that if $X_n \rightarrow X$ in measure, then $X_n \rightarrow X$ in distribution.

    *Show that an increasing real function on $\R$ can have at most countably many discontinuities. Let $X_n \rightarrow X$ in distribution. Show that there exists a probability space $(\Omega,\curF,\P)$ and random variables $\tilde{X}$, $\tilde{X_n}$ on $\Omega$ for all $n$, such that $X$ has the same distribution function as $\tilde{X}$ and $X_n$ has the same distribution function as $\tilde{X_n}$, and that $\tilde X_n \rightarrow \tilde X$ a.s..
\end{question}

\begin{question}{11}
    Let $(X_n)$ be a sequence of independent random variables. Define the tail $\sigma$-algebra of $(X_n)$. 

    State and prove the Kolmogorov's zero-one law.

    Let $F_{X_n}$ be the distribution function of $X_n$. Suppose $F_{X_n} = F$ for all $n$, and that $F(x) < 1$ for all $x\in \R$. Show that $(X_n)$ is unbounded above a.s.. Deduce that $\limsup_n X_n = \infty$ a.s..

    Suppose that $F(x) = 1 - e^{-x}$, show that $\limsup_n X_n/\log n =1$ a.s..
\end{question}

\section{Integration}
\begin{question}{12}
    What is a simple function $f$? What is the integral $\mu(f)$ of a simple function? Let $a_1, \dots, a_n$ be positive numbers, $A, A_1, \dots, A_n$ be sets, *show that if $1_A = \Sigma_{i=1}^n a_i 1_{A_i}$, then $\mu(A) = \Sigma_{i=1}
    ^n a_i \mu(A_i)$. Hence, show that $\mu$ is well-defined.
    
    
    Let $f,g$ be simple functions and $\alpha,\beta$ be positive constants, check the following properties of $\mu$:

    (i) $\mu(\alpha f + \beta g) = \alpha \mu(f) + \beta \mu(g)$.

    (ii) $f \leq g$ implies $\mu(f) \leq \mu(g)$.

    (iii) $f = 0$ a.e. if and only if $\mu(f) = 0$.

    \bigskip

    What is the integral for a non-negative measurable function? Show that this definition is consistent with the integral of a simple function. Let $f$ be a measurable function, when is $f$ integrable? Show that $|\mu(f)| \leq \mu(|f|)$ for all integrable functions $f$.
\end{question}

\begin{question}{13}
State and prove the monotone convergence theorem. Prove the theorem again by a different method. Deduce a sequential form of the theorem.

Show that for non-negative measurable functions $f$ and $f_n$, $f_n \uparrow f$ a.e. implies that $\mu(f_n) \uparrow \mu(f)$.

Let $f,g$ be non-negative measurable functions and $\alpha,\beta$ be non-negative constants, check the following properties of $\mu$:

    (i) $\mu(\alpha f + \beta g) = \alpha \mu(f) + \beta \mu(g)$.

    (ii) $f \leq g$ implies $\mu(f) \leq \mu(g)$.

    (iii) $f = 0$ a.e. if and only if $\mu(f) = 0$.

Let $f,g$ be integrable functions and $\alpha,\beta$ be real constants, check the following properties of $\mu$:

    (i) $\mu(\alpha f + \beta g) = \alpha \mu(f) + \beta \mu(g)$.

    (ii) $f \leq g$ implies $\mu(f) \leq \mu(g)$.

    (iii) $f = 0$ a.e. implies $\mu(f) = 0$.
\end{question}

\begin{question}{14}
    Let $\curA$ be a $\pi$-system containing $E$ which generates $\curE$ as a $\sigma$-algebra. Show that for any integrable function $f$, if $\mu(f 1_A) = 0$ for all $A \in \curA$, then $f = 0$ a.e.. (Hint: Dynkin's $\pi$-system lemma.)
\end{question}

\begin{question}{15}
    State and prove Fatou's lemma.

    State and prove dominated convergence theorem.
\end{question}

\begin{question}{16}
    Let $(E,\curE,\mu)$ be a measure space and let $A \in \curE$. Let $\curE_A$ be the collection of measurable sets contained in $A$. Show that $\curE_A$ is a $\sigma$-algebra on $A$ and that the restriction $\mu_A := \mu|_{\curE_A}$ is a measure on $\curE_A$. Moreover, show that for any non-negative measurable function $f$ on $E$, $\mu(f 1_A) = \mu_A(f|_A)$. Suppose $\mu(A) = 0$, show that the above equation is always zero. 

    Let $(E,\curE,\mu)$ be a measure space and $f$ be a non-negative measurable function on $E$. Define $\nu(A) = \mu(f 1_A)$ for all $A \in \curE$. Show that $\nu$ is a measure on $\curE$ and $\nu(g) = \mu(fg)$ for every non-negative measurable function $g$ on $E$.

\end{question}

\begin{question}{17}
    Let $(E,\curE)$ and $(G,\curG)$ be measurable spaces, $f:E \rightarrow G$ be measurable. Let $\mu$ be a measure on $\curE$, $\nu = \mu \circ f^{-1}$  be the image measure on $\curG$ induced by $f$. Show that for any non-negative measurable function $g$ on $G$, $\nu(g) = \mu(g \circ f)$.
    
\end{question}

\begin{question}{18}
    State and prove the fundamental theorem of calculus for continuous functions under the Lebesgue measure on $[a,b] \subset \R$.

    Let $\phi : [a,b] \rightarrow \R$ be a continuously differentiable and strictly increasing function. Show that for any non-negative measurable function $g : [\phi(a),\phi(b)] \rightarrow \R$, 
    \[
    \int_{\phi(a)}^{\phi(b)}g(y) dy = \int_a^b g(\phi(x)) \phi'(x) dx.
    \]
\end{question}

\begin{question}{19}
Let $U$ be an open set in $\R$, $(E,\curE,\mu)$ be a measure space. Let $f : U \times E \rightarrow \R$ such that $f$ is differentiable in the first entry and integrable in the second entry. State the condition when the following equation is well-defined and holds:
    \[
    \frac{d}{dt}\int_E f(t,x) \mu(dx) = \int_E \frac{\partial}{\partial t} f(t,x) \mu(dx).
    \]
Prove your statement.
\end{question}

\begin{question}{20}
    Let $(E_1,\curE_1)$ and $(E_2,\curE_2)$ be finite measurable spaces, define the product $\sigma$-algebra $\curE = \curE_1 \otimes \curE_2$ on $E = E_1 \times E_2$.

    Let $f$ be an $\curE$-measurable function, show that for all $x_1 \in E_1$, the function $x_2 \mapsto f(x_1,x_2)$ is $\curE_2$-measurable. (Hint: monotone class theorem.)

    Let $f$ be a bounded(non-negative resp.) $\curE$-measurable function, define $f_1(x_1) = \int_{E_2} f(x_1,x_2) \mu_2(dx_2)$, show that $f_1$ is a bounded(non-negative resp.) $\curE_2$-measurable function.
\end{question}

\begin{question}{21}
    Let $(E_1,\curE_1,\mu_1)$ and $(E_2,\curE_2,\mu_2)$ be two finite measure spaces. Let $E = E_1 \times E_2$ and $\curE = \curE_1 \otimes \curE_2$. Show that there exists a unique measure $\mu = \mu_1 \otimes \mu_2$ on $(E,\curE)$ such that $\mu(A_1 \times A_2) = \mu_1(A_1)\mu_2(A_2)$ for all $A_1 \in \curE_1$ and $A_2 \in \curE_2$.

    Let $\hat \curE = \curE_2 \otimes \curE_1$ and $\hat \mu = \mu_2 \otimes \mu_1$. Given $f: E_1 \times E_2 \rightarrow \R$, define $\hat f : E_2 \times E_1 \rightarrow \R$ by $\hat f (x_2,x_1) = f(x_1,x_2)$. Show that given $f$ non-negative and $\curE$-measurable, $\hat f$ is non-negative and $\hat \curE$ measurable such that $\mu(f) = \hat \mu(\hat f)$. 

    State and prove Fubini's theorem for non-negative $\curE$-measurable functions and for $\mu$-integrable functions.

    Let $X_1, \dots, X_n$ be random variables on probability space $(\Omega, \curF, \P)$ with values in $(E_1,\curE_1),\dots,(E_n,\curE_n)$ respectively. Let $E = E_1 \times \dots \times E_n$ and $\curE = \curE_1 \otimes \dots \otimes \curE_n$. Let $X(\omega) = (X_1(\omega),\dots,X_n(\omega))$ for all $\omega \in \Omega$. Show that $X$ is $\curE$-measurable. Show that the following are equivalent:

    (a) $X_1,\dots,X_n$ are independent.

    (b) $\mu_X = \mu_{X_1} \otimes \dots \otimes \mu_{X_n}$.

    (c) For all $f_k$ bounded $\curE_k$-measurable functions for $k = 1,\dots,n$, 
    \[
    \E(\prod_{k=1}^n f_k(X_k)) = \prod_{k=1}^n \E(f_k(X_k)).
    \]
\end{question}

\section{Norms and Inequalities}

\begin{question}{22}
	Let $(E,\curE,\mu)$ be a measure space. What is $L^p(E,\curE,\mu)$ for $p \in [1,\infty)$? What is $L^\infty(E,\curE,mu)$? When does $f_n$ converge to $f$ in $L^p$?
	
	State and prove Chebyshev's inequality. Deduce the tail estimate.
	
	State the supporting plane lemma for a real convex function on a interval. State and prove Jensen's inequality for integrable random variable $X$. Show that each term used is well-defined.
\end{question}

\begin{question}{23}
	State and prove Holder's inequality of measurable functions $f$ and $g$ by creating a density measure about $f$.
	
	State and prove Minkowski's inequality.
	
	Let $\curA$ be a $\pi$-system generating $\curE$ and $\mu(A) < \infty$ for all $A \in \curA$ and there exists $E_n \in \curA$ such that $E_n \uparrow E$.  Show that the algebraic span of $\{1_A\}_{A\in \curA}$ is dense in $\curL^p$, i.e. for every $f \in L^p$ and for every $\epsilon > 0$, there exists $v \in \text{span}\{1_A\}_{A\in \curA}$ such that $||f - v||_p < \epsilon$.
\end{question}

\section{Completeness of $L^p$ and Orthogonal Projection}
\begin{question}{24}
	Show that $L^p$ is complete for $p = \infty$ and $p \in [1,\infty)$, i.e. given sequence $(f_n)$ in $L^p$ with $||f_n - f_m||_p \rightarrow 0$ as $n,m \rightarrow \infty$, there exists $f \in L^p$ such that $||f - f_n|| \rightarrow 0$.
\end{question}

\begin{question}{25}
	Let $V$ be a closed subspace of $L^2$, then for every $f \in L^2$, there exist $v \in V$ and $u \in V^\perp$ such that $f = v + u$ and $||f-v||_2 \leq ||f-g||_2$ for all $g \in V$, with equality holds if and only if $g = v$ a.e..
\end{question}
 \end{document}